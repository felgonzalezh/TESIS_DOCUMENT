\chapter[Physics Object Reconstruction]{Physics Object Reconstruction}
\label{chap:ParticleID}
%PONER PARRAFO de introduccion al capitulo.
%fo
%PF jet reconstruction is an important key for the search of $Z'$ bosons in the di-hadronic tau channel since 
%jets are used as initial seed for the tau-lepton reconstruction. \\

%The energy of a jet is carried by charged hadrons ($~65 \%$), photons 
%from $\pi^{0}$ decays ($~15 \%$) and neutral hadrons ($~20 \%$) \cite{CMS-PAS-PFT-10-001}. \\


\section{Particle Flow}
\label{sec:PF}

\subsection{Track Reconstruction}
\label{subsec:TrackReco}

\subsection{Vertex Reconstruction}
\label{subsec:VertexReco}

\subsection{Clustering}
\label{subsec:Clustering}

\subsection{Link algorithm}
\label{subsec:Linkalgorithm}

\section{Jets Reconstruction}
\label{sec:Jet}

\section{b-jet Identification}
\label{sec:bJet}

\section{MET}
\label{sec:MET}

\section{Muon Identification}
\label{sec:Muon}


\section{Electron Identification}
\label{sec:Electron}



\section{Tau Reconstruction and Identification}
\label{sec:Tau}

\subsection{Tau Reconstruction Algorithm}
\label{subsec:HPS}

\subsection{Tau Discriminators}
\label{subsec:Discriminators}

\subsection{Performance of Tau Reconstruction Algorithm}
\label{subsec:Performance}

\subsection{Tau Reconstruction Changes for Run II}
\label{subsec:Changes}

%The $\vec{\not{E}_{T}}$ is corrected taking into account those corrections performed to the jet $p_{T}$:

%CMS can be considered as an hermetic detector since almost all the final states 
%coming from a pp collision can be measured and identified, with exception
%of neutral particles that only interacts weakly with matter such as 
%the neutrino and the hypothetical neutralino. Although these particles 
%scape from CMS without detection, it is possible to infer the 
%transverse momenta carried by all of them making use of the 
%transverse momentum conservation. The missing transverse momentum ($E_{T}^{miss}$),
%the momentum (in the orthogonal plane to the beam line) carried by all particles that only interact weakly, is defined
%as the momentum imbalance in the transverse plane to the beam line of all
%particles detected. 



%excluding the region close to the beam
%pipe and the gaps between the various subdetectors and between the di↵erent modules of each



%An efficiency larger than 80% is obtained for jets with a p T > 20 GeV/c. The 100% plateau is reached above
%40 GeV/c, at which point the mismatched jet rate is negligible





%JER Article
%The jet p T resolutions are determined with both dijet and photon+jet events, as discussed in
%section 8. The reference resolutions obtained from simulation are parameterized as a function of
%particle-level jet p T, ptcl (defined in section 2) and average number μ of pileup interactions in bins
%of jet η. Corrections for differences between data and MC simulation are applied as η-binned scale
%factors




%Since in average 85 $\%$ of the constituents of a jet are charged particles and photons, the jet energy resolution

%PF jet momentum and spatial resolutions are greatly improved with respect to calorimeter jets, as
%the use of the tracking detectors and high granularity of the ECAL improves the energy resolution
%through the independent measurements of charged hadrons and photons inside a jet, which together
%2.1constitute ≈85% of the average jet energy. In reconstructing the PF candidate four-momentum,
%photons are assumed massless and charged hadrons are assigned the charged pion mass.

%As mentioned previously, the typical jet energy fractions carried by charged particles, photons
%and neutral hadrons are 65%, 25% and 10% respectively. These fractions ensure that 90% of
%the jet energy can be reconstructed with good precision by the particle-flow algorithm, both in
%value and direction, while only 10% of the energy is affected by the poor hadron calorimeter
%resolution and by calibration corrections of the order of 10 to 20%. As a natural consequence,
%it is expected that jets made of reconstructed particles be much closer to jets made of MonteCarlo–generated
%particles than jets made from the sole calorimeter information, in energy, direction
%and content. It is the purpose of this section to quantify this statement.

%($c\tau > 1$ cm)


%\subsection{Online Identification}
%\label{subsec:JetTrigger}

%\subsection{Offline Identification}
%\label{subsec:JetReconstruction}



%\subsection{Online Identification}
%\label{subsec:ElectronTrigger}

%\subsection{Offline Identification}
%\label{subsec:ElectronReconstruction}

%%%% DEFINED


%\section{Photon Reconstruction}
%\label{sec:Photon}


%\section{Muon Reconstruction}
%\label{sec:Muon}

%\section{Electron Reconstruction}
%\label{sec:Electron}

%\section{B-Jet Reconstruction}
%\label{sec:BJet}



%\section{Tau Lepton}
%\label{sec:Tau}

%\subsection{Tau Reconstruction}
%\label{subsec:TauTrigger}

%\subsection{Tau Reconstruction}
%\label{subsec:TauReconstruction}

%\subsection{Working Points}
%\label{subsec:wp}

%\subsubsection{Efficiency of Working Points}
%\label{subsubsec:Eff_WP}

%\subsection{Fake Rates}
%\label{subsec:FakeRates}

%\subsection{Perspectives Run III}
%\label{subsec:Perspectives} 