\chapter*{Introduction}
\addcontentsline{toc}{chapter}{Introduction}

Our understanding of the behavior of elementary particles and the interactions between them is 
currently described by the Standard Model (SM). Nevertheless, there are still open 
questions in particle physics that are not addressed by the SM; for instance, SM does 
not provide an explanation for the matter-antimatter asymmetry, it does not explain the origin of neutrino's masses as well as the 
neutrino oscillations; besides SM does not predict the nature of dark matter and dark energy,
it does not include a quantum version of gravity, among others. For these reasons, 
many theories, known as Beyond Standard Model (BSM) \cite{BSM}, have been proposed by extending the SM symmetry group
in order to address these still open questions. \\

One of the most simplest ways of the extensions of the Standard Model (SM) is through an additional
U(1)$^{\prime}$ symmetry group, which would give rise to the existence of a heavy neutral gauge boson, referred as \Zprime. These 
kind of models were initially motivated by the electroweak symmetry breaking since it leaves to the unbroken group 
U(1)$_{\text{EM}}$; then, in a similar way, a U(1)$^{\prime}$ group might be a remnant of breaking  
an extended SM symmetry. However, the most important motivation for U(1)$^{\prime}$ symmetries 
and its associated \Zprime~is involved with the Grand Unification Theories (GUT) since, in some scenarios, the \Zprime~bosons
would have masses at TeV scale and consequently, if they exist, they would be produced at the Large Hadron 
Collider (LHC) at CERN \cite{Langacker:2008yv}. Besides the excitement of a new gauge boson discovery, the existence 
of a \Zprime~would stand for a clear evidence of Physics Beyond SM which would come from scenarios of 
Supersymmetry (SUSY) or Superstring Models \cite{Langacker:2008yv}. \\

Because of the \Zprime~bosons are predicted at TeV scales and they would couple to the SM fermions, 
these hypothetical particles would be produced in the proton-proton collisions at the LHC; and the \Zprime~ 
signature would be reflected as a heavy resonance in the invariant mass distribution of its decay products 
(opposite-charged fermions). During the last two decades, the evidence of the \Zprime~boson 
has been excluded by several searches: The CDF and D0 experiments at the Tevatron, and the CMS and ATLAS experiments 
at the LHC have constrained the existence of \Zprime~bosons in a wide range of mass. The tightest 
upper limits on the \Zprime~mass, for the most simplified model, are 3.2 TeV and 3.4 TeV which were 
settled (respectively) by the searches performed by CMS and ATLAS experiments in the dilepton final 
state (Z$'\rightarrow \ell\ell$, where $\ell=e, \mu$) using 2015 Data \cite{CMSZprime2dileptonbib,ATLASZprime2dileptonbib}. \\

The most simplified model is the Sequential Standard Model (SSM), where the \Zprime~bosons 
obey (inspired by the $Z^{0}$ SM gauge boson) the universality of the SM couplings; for this reason, 
it is used as a brenchmark model for the \Zprime~searches. Nevertheless, there are 
also many BSM scenarios which predict a generational coupling dependence where the \Zprime~boson 
would decay preferentially to the third generation of fermions; for instance, 
the topcolor-assisted technicolor (TAT) models, which try to explain the high mass of the top quark,
predicts a heavy gauge boson \ZprimeTAT~with enhanced couplings, that would decay mainly 
to the third generation \cite{Langacker:2008yv}. These models are an especial compelling motivation to 
search for \Zprime~bosons decaying into taus pairs: the purpose of this work. Consequently, if a resonance 
were found in the other fermion-pair final states (such as $\ell\ell$), the search for \Zprimetotautau is also 
interesting since it would confirm or it would discard the universality of the couplings. \\

The search for \Zprime~decaying into tau-pairs usually involves four channels since 
the $\tau$ lepton has different decay modes: it can decay leptonically into a lighter lepton and two 
neutrinos (\taue, \taumu) or hadronically into pions and one neutrino (\tauh); therefore, the 
search for \Zprimetotautau includes the channels: \tauell\tauh, \taue\taumu and \tauh\tauh. During the Run I at LHC, CMS and ATLAS 
collaborations have performed searches for heavy resonances in the ditau final states, excluding it
for masses below 1.4 TeV for the SSM \cite{CMSZprime2ditaubib,ATLASZprime2ditaubib}. Since Run II started, CMS 
collaboration has performed also the search using 2015 data samples from proton-proton collisions 
at a center-of-mass energy of 13 TeV. CMS collaboration reported a new upper limit of 1.7 TeV for the
\Zprime~mass \cite{CMS_Zprime2tays2015}. Nevertheless, the luminosity of 2016 data have increased considerably, making 
possible a new attempt of the search for \Zprime~bosons decaying tau-pairs. In this document is presented the search for new 
heavy neutral particles decaying into two hadronic taus, using the 2016 data collected by the CMS experiment
from proton-proton collisions at a center-of-mass energy of 13 TeV.\\

%Althought the \Zprimetotauh has the highest expected branching ratio, this 
%channel has a copious contribution of background from (QCD) multijet production, making interesting 
%this channel from the experimental point of view. 

This document is organized as follows: The Physics of the \Zprime~boson, including a theorical introduction
and previous experimental searches, is presented in Chapter \ref{chap:Zp}. The experimental apparatus such as the LHC and the CMS Detector 
which is needed to record all the data are described in Chapter \ref{chap:CMSExp}. A detailed description of the physics objects reconstructed 
by CMS, focused in the understanting of tau algorithms, is presented in Chapter \ref{chap:ParticleID}. All experimental techniques  
involved in the search for \Zprime~using 2016 data is described in Chapter \ref{chap:AnalysisStrategy}. Finally, Chapter \ref{chap:Analysis} shows 
the studies and results of the proposed search.

