\chapter*{Introduction}
\addcontentsline{toc}{chapter}{Introduction}

Our understanding of the behavior of elementary particles and the interactions between them is 
currently described by the Standard Model (SM). Nevertheless, there are still open 
questions in particle physics that are not addressed by the SM; for instance, SM does 
not provide an explanation for the matter-antimatter asymmetry, it does not explain the origin of neutrino's masses as well as the 
neutrino oscillations; besides SM does not predict the nature of dark matter and dark energy,
it does not include the a quantum version of gravity, among others. For these reasons, 
many theories, known as Beyond Standard Model (BSM) \cite{BSM}, have been proposed in order to address these still open questions. \\

One of the most simplest ways of the extensions of the Standard Model (SM) is through an additional
U(1)$^{\prime}$ symmetry group, which would give rise to the existence of the \Zprime gauge boson. These 
kind of models were initially motivated by the electroweak symmetry breaking that leaves to the unbroken group 
U(1)$_{\text{EM}}$, since in a similar way, a U(1)$^{\prime}$ unbroken group might be a remnant of breaking  
an extended SM symmetry. Nevertheless, the most important motivation for U(1)$^{\prime}$ symmetries 
and its associated \Zprime is involved with the Grand Unification Theories (GUT) because in some scenarios the \Zprime bosons
would have masses at TeV scale and then, if they exist, they would be produced at the Large Hadron Collider (LHC) at CERN \cite{Langacker:2008yv}. \\

The \Zprime is a heavy neutral gauge boson which would couple to the SM fermions. Due to this reason and the fact that it is predicted 
in the TeV scale, \Zprime would be produced in the proton-proton collisions at the LHC and it would be detected in its multi-purpose detectors 
CMS and ATLAS as a heavy resonance in the invariant mass distribution of its decay products (opposite-charged fermions). Along with
excitement of a new gauge boson discovery, the existence of a \Zprime  would stand for a clear evidence of Physics
Beyond SM which would come from scenarios of Supersymmetry (SUSY) or Superstring Models. \\




%The \Zprime bosons is predicted in the TeV scale by GUT such as SO(10) and E$_{6}$, Supersymmetry (SUSY) models and Superstring models.

%the existence of Z$'$ bosons in the scale of the TeV \cite{Langacker:2009su}. 

%Since a Z$'$ boson would couple to the SM fermions, it might be produced at hadron colliders, and might be observed as a resonance
% in the invariant mass distribution of its decay products: opposite-charged fermions. The identification
% of one of those massive resonances would be a clear evidence of physics BSM; however, the mass of the Z$'$
%bosons is a free parameter in the theoretical models, and therefore its experimental search
%must be performed scanning a wide range of energies.



