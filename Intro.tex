\chapter*{Introduction}
\addcontentsline{toc}{chapter}{Introduction}

Our understanding of the behavior of elementary particles and their interactions is 
currently described by the Standard Model (SM). Nevertheless, there are still open 
questions in particle physics that are not addressed by this model. For instance: the SM does 
not provide an explanation for the matter-antimatter asymmetry; it does not 
explain the origin of the neutrino's masses, as well as the 
neutrino oscillations; besides, the SM does not explain the nature of dark matter and dark energy;
it does not include a quantum version of gravity; among others. For these reasons, 
many theories, known as Beyond Standard Model (BSM) \cite{BSM} theories, have been proposed.
Most of them extend the SM symmetry group in order to address the open questions. \\

One of the simplest ways of extending the SM symmetry is through an additional
\Uprime~group. This symmetry would give rise to the existence of a new neutral gauge boson, generically 
referred as \Zprime. These kind of models were initially motivated by the 
SM electroweak symmetry breaking, since it leaves the U(1)$_{\text{EM}}$ group 
as a remnant. In a similar way, breaking an extended SM symmetry could resort 
in an extra \Uprime~symmetry group. The additional \Uprime~group and its associated
\Zprime~appear in several scenarios of Grand Unification Theories (GUT), Super Symmetry 
(SUSY) and Super String. In the GUT case, at least one \Zprime~boson
is predicted for scenarios where the gauge group is greater than SU(5), for instance 
SO(10) or E$_{6}$. There are some SUSY and Super String versions of GUT 
where the SM electroweak and the \Uprime~symmetry groups 
generally break at the soft SUSY breaking scale \cite{Langacker:2008yv}; in these scenarios 
the \Zprime~boson would have a mass in the TeV range and, consequently, its experimental
observation would stand for a clear evidence of Physics Beyond SM, which could 
come from scenarios of Super Symmetry or Super String Models. \\

The discovery of a massive \Zprime~boson would have profound implications due to the nature of 
the \Uprime~symmetry breaking, since it would require an extended Higgs sector and, in the case 
of SUSY scenarios, it would also require an extended neutralino sector \cite{Langacker:2008yv}. This 
would have consequences for particle physics and for cosmology related with the dark matter problem. For 
example some SUSY models that predict a \Zprime, also predict extra Higgs, a right-handed neutrino and their 
superpartners; in this scenarios the right-handed sneutrino is a candidate for dark matter due to 
its interactions with the \Zprime~boson \cite{TheorySUSY_DM}. Besides, \Zprime~discovery would also 
have implications for the nature of the neutrino mass, for the nature of the SUSY 
hidden sector and for the possible mediator(s) of SUSY breaking. For instance, some SUSY scenarios
constrain the \Uprime~group because of a possible exotic particles production from the \Zprime~decays; this 
constraints are involved with the neutrino mass prediction. On the other hand, there are many BSM models which attempt to solve the problem of 
quadratic divergencies for  the Higgs mass through an extra U(1)$'$ symmetry, 
for example models with \emph{Little Higgs} \cite{Langacker:2008yv}. The \emph{Little Higgs} models would
can produce in new Higgs in the TeV energy scale as well as one or more \Zprime~bosons. \\

The \Zprime~searches are motivated by those BSM scenarios which predict \Zprime~bosons with masses at TeV scale 
and consequently, if they exist, they would be produced at the Large Hadron Collider (LHC) at CERN \cite{Langacker:2008yv}. 
In the \Zprime~searches the theoretical model generally used as a benchmark is the Sequential Standard Model (SSM). 
This is the simplest model since, the \ZprimeSSM~boson obeys the universality of the SM couplings: it has 
the identical couplings than the SM $Z^{0}$ gauge boson. Nevertheless, there are  
also many BSM scenarios that predict a generational coupling dependence, where the \Zprime~boson 
would decay preferentially to the third generation of fermions \cite{ZprimeThirdGeneration}. For instance, the topcolor-assisted technicolor (TAT) models, which attempt to give an explanation to the high mass of the top quark,
predict a heavy gauge boson \ZprimeTAT~with enhanced couplings to the third generation \cite{TAT}. These models 
are an especial compelling motivation to search for \Zprime~bosons decaying into tau pairs, which is the main goal of this PhD 
dissertation. Consequently, if a resonance were found in the other fermion-pair final states (such as $\ell\ell$), the search for 
\Zprimetotautau~would be very important since it would confirm or it would discard 
the hypothesis of the universality of the couplings. \\

%Nevertheless , in some scenarios 
%the \Zprime~would have family-dependent charges, which would lead to Flavor-Changing Neutral Current 
%contributions (FCNC) \cite{Langacker:2008yv}. 

Because of the \Zprime~bosons are predicted at TeV scales and they would couple to the SM fermions, 
these hypothetical particles would be produced in the proton-proton collisions at the LHC. The \Zprime~ 
signature would be reflected as a heavy resonance in the invariant mass distribution of its 
decay products: opposite-charged fermions (i.e. high momentum leptons or jets coming from $q\bar{q}$). During 
the last two decades, the evidence of a \Zprime~boson has been excluded by several searches. 
The CDF \cite{CDFZprimedielectronbib,CDFZprimedimuonbib,CDFZprimeditaubib,CDFZprimeditopbib}  
and D0 \cite{D0Zprimesearchesbib,D0Zprimetodielectronbib,D0Zprimeditopbib} experiments at the Tevatron, and
the CMS \cite{CMSZprime2dileptonbib,CMSZprime2ditaubib,CMSZprime2tausRunII,CMSZprime2ditauelectronmuonbib,CMSZprime2toptop,CMSZprime2bbbib,CMSZprime2dijetbib}
and ATLAS \cite{ATLASZprime2dileptonbib,ATLASZprime2ditaubib,ATLASZprime2toptopbib, ATLAS_Zprime2tausRunII} experiments at the LHC, have 
constrained the existence of \Zprime~bosons in a wide range of mass. The tightest 
upper limits on the \Zprime~mass have been set by the CMS and ATLAS experiments in the dilepton final 
state (Z$'\rightarrow \ell\ell$, where $\ell=e, \mu$), where the \ZprimeSSM~has been excluded below 3.2 TeV (CMS) and 3.4 TeV (ATLAS). These 
searches were performed with the data collected by the experiments during 2015, for proton-proton 
collisions at a \centermassenergy~of 13 TeV \textbf{\textcolor{red}{REFERENCIA OJO Articulos Dilepton RunII}}. The channels
where the \Zprime~decays into a light lepton pairs ($ee$ and $\mu\mu$ final states) have been explored
widely due to their high mass resolution, large acceptance and relative low background. The most significative 
background in the dilepton channels comes from Drell-Yan (DY) processes, which have been studied extensively. This 
background has the same topology as the hypothetical \Zprime~events, but it can be largely 
suppressed since its reconstructed mass lays mainly around the Z boson mass. Nevertheless, the 
reconstructed mass distribution of DY+jets events has a long tail extending to large values, which 
makes them an important and irreducible background for \Zprime~searches.  \\

The experimental signature of the \Zprimetotautau~channel consists on events with high \pt, almost
back-to-back and oppositely charged $\tau$-pairs. The search for \Zprime~decaying into tau-pairs 
usually involves four experimental signatures, since the $\tau$ lepton has different decay modes: it 
can decay leptonically into a lighter lepton and two neutrinos (we will refer to the taus that decay in 
this way as \taue~or \taumu, depending on the lepton in which they decay); 
or it can decay hadronically into a set of pions and a neutrino (we will refer to the taus that decay in this form
as \tauh). Therefore, the search for \Zprimetotautau~includes 
the channels: \taue\tauh, \taumu\tauh, \taue\taumu~and \tauh\tauh. The \taue\taue~and \taumu\taumu~channels
are not included in the search due to their relative low branching ratio, and because of the difficulty to
distinguish their signature from the dilepton channels (\Zprimetomumu, \Zprimetoee). The $\tau$ 
identification represents an important experimental challenge for several reasons: first, since 
the $\tau$ can not be reconstructed fully due to neutrinos, the momentum can not be known; additionally, the leptonic decay of 
the tau can not be identified since it has the same signature than events in which leptons were produced as a result of 
a collision; and finally, the hadronic decay has a similiar signature to a QCD jet. Therefore, sofisticated tau 
identification algorithms have been developed by CMS and ATLAS experiments in order improve the identification 
efficiency. The CMS experiment uses the Hadron Plus Strips (HPS) algorithm, that 
identifies the hadronic tau leptons with an efficiency of 60 $\%$. Therefore, the difficulty of the tau 
identification, makes the search of \Zprimetotautau~challenging. \\

The \Zprime$\rightarrow$\tauell\tauh~are the most sensitive channels since they have 
a relative high yield, a better acceptance due to the presence of light leptons, and a relatively
low QCD background. The \Zprimetotauh~channel has the highest 
expected yield but it also has a very high background coming 
from QCD multijet production. The purpose of this dissertation is to search for \Zprime~bosons decaying into two hadronic taus 
in proton-proton collision at a \centermassenergy~of 13 TeV, with the data collected by the 
CMS  experiment during 2016. \\

% In the CMS collaboration, the \Zprime~searches 
%are managed by the Exotic group, meanwhile the searches for \Zprimetotautau~have 
%being performed by the BSM3G Group, which I have been collaborating from 2015. (MEJOR DESCRIPCION DE ESTO, LOS GRUPOS DE CERN)\\

%, making interesting this channel from the experimental point of view. 

During the Run I at the LHC, the CMS and ATLAS experiments performed searches for heavy resonances 
in the ditau final states, using the combined channels mentioned above \cite{CMSZprime2ditaubib,ATLASZprime2ditaubib}.
CMS has excluded the \ZprimeSSM~for masses below 1.4 TeV, using data collisions at \sqrts 7 TeV; while ATLAS 
has set an exclusion limit for the \ZprimeSSM~mass of 2.02 TeV, using data at collisions \sqrts 8 TeV. 
CMS Collaboration has also carried out a search for a \Zprime$\rightarrow$\taue\taumu~boson, using collisions at \sqrts 8 TeV, excluding
the \ZprimeSSM~for masses below 1.3 TeV \cite{CMSZprime2ditauelectronmuonbib}. With the high luminosity and the new energy range 
reached by the LHC during Run II, CMS has performed also the \Zprimetotautau~search at \sqrts 13 TeV using the 2015 data, and reporting 
%a new upper limit of 1.7 TeV for the \ZprimeSSM~mass \cite{CMS_Zprime2tays2015}. On the other hand, ATLAS experiment
a new upper limit of 2.1 TeV for the \ZprimeSSM~mass \cite{CMSZprime2tausRunII}. The higher luminosity reached 
by the LHC during 2016 made possible a improved search. This document presents this search, using 
the data collected by the CMS experiment during 2016 at a \centermassenergy of 13 TeV. \textbf{\textcolor{red}{REGRESAR AL FINAL A PONER LOS RESULTADOS}}\\
%We expect to exclude the \ZprimeSSM~boson decaying into two hadronic taus for masses 
%below 2.7 TeV, but combining this result with the remaining ditau channel analysis (performed by the other members of the BSM3G group)
%we expect to exclude it for masses below of 3.0 TeV. The combined result improves the sensitivite of the previous search. \\


This document is organized as follows: in Chapter \ref{chap:Zp} I present the theoretical bases of physics of 
the \Zprime~boson, presenting the most relevant models as well as the results for previous searches. In 
Chapter \ref{chap:CMSExp}, all CMS experiment is described. I start with a brief description 
of the LHC, and afterwards a full description of the CMS detector and its subdetectors, making emphasis 
in the tau detection. Since the tau identification is crucial in order to distinguish the \Zprime~events, in Chapter \ref{chap:ParticleID}
I present a detailed description of the physics objects reconstruction algorithms used
by CMS collaboration, focusing mainly in tau algorithms. All the experimental techniques involved in the 
identification of the \Zprimetotauh~signature are presented in Chapter \ref{chap:AnalysisStrategy}. 
I discuss all the backgrounds for this channel and explain the all techniques that I have developed 
to estimate them. Finally, the results of the search and discussions about additional studies performed to identify the \Zprime~ 
are presented in the Chapter \ref{chap:Analysis}.


%Although, the signature of a \Zprime would be a heavy resonance in the invariant mass distribution, and in consequence the experimental 
%search of this particles is model-independent, the 
