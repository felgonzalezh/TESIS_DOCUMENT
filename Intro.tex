\chapter*{Introduction}
\addcontentsline{toc}{chapter}{Introduction}

Our understanding of the behavior of elementary particles and the interactions between them is 
currently described by the Standard Model (SM). Nevertheless, there are still open 
questions in particle physics that are not addressed by the SM; for instance, SM does 
not provide an explanation for the matter-antimatter asymmetry, it does not explain the origin of neutrino's masses as well as the 
neutrino oscillations; besides SM does not predict the nature of dark matter and dark energy,
it does not include the a quantum version of gravity, among others. For these reasons, 
many theories, known as Beyond Standard Model (BSM) \cite{BSM}, have been proposed in order to address these still open questions. \\

One of the most simplest ways of the extensions of the Standard Model (SM) is through an additional
U(1)$^{\prime}$ symmetry group, which would give rise to the existence of the neutral gauge boson, referred as \Zprime. These 
kind of models were initially motivated by the electroweak symmetry breaking since it leaves to the unbroken group 
U(1)$_{\text{EM}}$; then, in a similar way, a U(1)$^{\prime}$ group might be a remnant of breaking  
an extended SM symmetry. Nevertheless, the most important motivation for U(1)$^{\prime}$ symmetries 
and its associated \Zprime is involved with the Grand Unification Theories (GUT) since, in some scenarios, the \Zprime bosons
would have masses at TeV scale and consequently, if they exist, they would be produced at the Large Hadron 
Collider (LHC) at CERN \cite{Langacker:2008yv}. Besides the excitement of a new gauge boson discovery, the existence 
of a \Zprime  would stand for a clear evidence of Physics Beyond SM which would come from scenarios of 
Supersymmetry (SUSY) or Superstring Models \cite{Langacker:2008yv}. \\

%The \Zprime is a heavy neutral gauge boson which would couple to the SM fermions; 
Since the \Zprime bosons are predicted at TeV scales and since they would couple to the SM fermions, 
these hypothetical particles would be produced in the proton-proton collisions at the LHC.
Nevertheless, there are also models where 
The most simplified model is the Sequential Standard Model (SSM), where the \Zprime bosons 
obey, inspired by the $Z^{0}$ SM gauge boson, the universality of the couplings to the SM. 


The \Zprime is a heavy neutral gauge boson which would couple to the SM fermions. Due to this reason and the fact that it is predicted 
in the TeV scale, \Zprime would be produced 
 and it would be detected in its multi-purpose detectors 
CMS and ATLAS as a heavy resonance in the invariant mass distribution of its decay products (opposite-charged fermions). 



The searches for \Zprime bosons in the proton-proton collisions at the LHC are possible since this hypo



There are two reason that leaves us to search for \Zprime bosons 


The \Zprime is a neutral gauge boson. Their couplings with the SM fermions and since

In the most simplified model, the \Zprime bosons obeys, inspired by the $Z^{0}$ SM gauge boson, the
universality of the couplings to the SM. 
which would couple to the SM fermions and since the prediction 
of the \Zprime bosons at TeV scales and the fact

Usually, the
mass limits in the searches for \Zprime bosons are referred to the most simplified model where, inspired by the $Z^{0}$ SM gauge boson, the
\Zprime obeys the universality of the couplings. Nevertheless, many models predict a generational coupling dependence 



with an additional U(1)$^{\prime}$ obey, inspired


is the Sequential Standard Model (SSM)


The nature of the couplings




Then searching for \Zprime bosons is particularly 
interesting since


The \Zprime is a heavy neutral gauge boson which would couple to the SM fermions and since the prediction 
of the \Zprime bosons at TeV scales and the fact 

The \Zprime is a heavy neutral gauge boson which would couple to the SM fermions. Due to this reason and the fact that it is predicted 
in the TeV scale, \Zprime would be produced in the proton-proton collisions at the LHC and it would be detected in its multi-purpose detectors 
CMS and ATLAS as a heavy resonance in the invariant mass distribution of its decay products (opposite-charged fermions). 

Along with excitement of a new gauge boson discovery, the existence of a \Zprime  would stand for a clear evidence of Physics
Beyond SM which would come from scenarios of Supersymmetry (SUSY) or Superstring Models \cite{Langacker:2008yv}. 

The evidence of the \Zprime boson has been excluded by several searches performed during the last two decades. The 
CDF and D0 experiments at the Tevatron, and the CMS and ATLAS experiments at the LHC have  constrained the 
existence of \Zprime bosons in a wide range of mass. The tightest upper limits on the \Zprime mass, for the most simplified model, are
3.2 TeV and 3.4 TeV which were settled (respectively) by the searches performed by CMS and ATLAS experiments in dilepton final 
state (Z$'\rightarrow \ell\ell$, where $\ell=e, \mu$) \cite{CMSZprime2dileptonbib,ATLASZprime2dileptonbib}. 


Usually, the
mass limits in the searches for \Zprime bosons are referred to the most simplified model where, inspired by the $Z^{0}$ SM gauge boson, the
\Zprime obeys the universality of the couplings. Nevertheless, many models predict a generational coupling dependence 

the Z$'$ couples mainly to the third generation fermions


ese searches 
are performed using the most simplified model as a benchmark 

Nevertheless, the most simplified model obey to the universality of the couplings, inspired by the $Z^{0}$ SM gauge boson. 

These limits 
are for the most simplified model that predicts the \Zprime boson




additionally to the most simplest models 

the most simplified model as well as many of the BSM that predict \Zprime bosons obey the universality 
of the couplings but there are 


%The search for \Zprime bosons have been continuing during Run II

Of particular interest for this analysis note are models that include an extra neutral gauge bo-
son that decays to pairs of high-p T τ leptons. In particular, extensions to the SM proposed as an
explanation for the high mass of the top quark predict Z 0 bosons that typically couple to third-
generation fermions. Examples are the topcolor-assisted technicolor (TAT) models, where the
0
Z 0 is denoted as Z TAT
Additionally, models with enhanced Z 0 couplings to third-generation
fermions exist to explain the B-meson anomalies observed at LHCb. Although many models
with extra gauge bosons obey the universality of the couplings, models with generational de-
pendent couplings resulting in extra neutral gauge bosons that preferentially decay to τ leptons
make this analysis an important mode for discovery.

consists in 


Nevertheless, some BSM scenarios propose a Z$'$ boson
whose couplings depend on the fermion generations. In these models,
the Z$'$ couples mainly to the third generation fermions and, as a result,
it would decay copiously into tau pairs \cite{Langacker:2009su}, making this
an interesting channel to perform a search. Since the tau can decay
leptonically into a lighter charged lepton and two neutrinos ($\tau_{e}$, $\tau_{\mu}$), or
hadronically into a neutrino plus pions ($\tau_{h}$), a search of a Z$'$ boson
includes the Z$'\rightarrow\tau_{h}\tau_{h}$, Z$'\rightarrow\tau_{h}\tau_{\mu}$,
 Z$'\rightarrow\tau_{h}\tau_{e}$ and Z$'\rightarrow\tau_{e}\tau_{\mu}$ decay modes. 
 
 The channels Z$'\rightarrow\tau_{\mu}\tau_{\mu}$ and Z$'\rightarrow\tau_{e}\tau_{e}$ are
 not commonly used to search for a Z$'$ boson since they have the lowest expected branching fraction  ($6\%$ for both channels),
 and there is a considerable background contribution from Drell-Yan production. On the other hand,
 the channel Z$'\rightarrow\tau_{h}\tau_{h}$ has the largest branching fraction, but
 it has a copious contribution of background from production of events resulting from
 Quantum ChromoDynamics (QCD) multijet production. Therefore,
 the Z$'\rightarrow\tau_{h}\tau_{\mu}$ channel is one of the most interesting
 channels to search a Z$'$. Searches for a Z$'$ decaying
  into tau pairs, in these four channels, have been performed by CMS and ATLAS
experiments, using data samples from proton-proton collisions at a center-of-mass energy   %%%%%%%%%%%%%% VER SI ESCRIBO CENTER O CENTRE
of 7 and 8 TeV, produced during the first run of the LHC. The CMS and ATLAS collaborations
reported no evidence of a Z$'$ boson, and excluded its existence for
masses below 1.4 TeV in the most simplified model \cite{CMSZprime2ditaubib,ATLASZprime2ditaubib}. \\

However, with the energy range and the high luminosity that the LHC will reach during the second run,
the CMS detector can explore new kinematical regions, making possible
the search for Z$'$ bosons in a wider range of energy. The purpose of my Ph.D. dissertation
is to search for Z$'$ bosons decaying into two taus, where one tau
decays leptonically into a muon and the other one decays into hadrons (Z$'\rightarrow \tau_{\mu}\tau_{h}$). The search
will be performed using a data sample coming from proton-proton collisions at a center-of-mass energy
of 13 TeV collected by the CMS experiment during the second run of the LHC. \\

This document is organized as follows: a state of the art for the Z$'$ boson physics is presented in section
\ref{StateofArt}. The CMS experiment and its detectors are described in section \ref{ExperimentalAparatus}.
A detailed description of the search strategy for Z$'$ bosons is discussed in the section \ref{ExperimentalStrategy}.
The goals and the timetable of the proposed research are presented in sections \ref{Goals} and \ref{Timetable}, respectively.



%The \Zprime bosons is predicted in the TeV scale by GUT such as SO(10) and E$_{6}$, Supersymmetry (SUSY) models and Superstring models.

%the existence of Z$'$ bosons in the scale of the TeV \cite{Langacker:2009su}. 

%Since a Z$'$ boson would couple to the SM fermions, it might be produced at hadron colliders, and might be observed as a resonance
% in the invariant mass distribution of its decay products: opposite-charged fermions. The identification
% of one of those massive resonances would be a clear evidence of physics BSM; however, the mass of the Z$'$
%bosons is a free parameter in the theoretical models, and therefore its experimental search
%must be performed scanning a wide range of energies.



