\chapter[Experimental Setup]{The CMS experiment}
\label{chap:CMSExp}
%During the Run II, the Large Hadron Collider (LHC) has produced 
%produced proton-proton collisions at an unprecedented ~\centermassenergy, 
%and at a high luminosity never reached before for 
The CMS and ATLAS experiments are the biggest multi-purpose particle detectors
ever built in the world and  are part of the LHC \cite{chp2:LHCTDR}. The LHC is a 
proton collider located at the CERN laboratory in Switzerland. The 
unprecedented \centermassenergy~of the proton-proton collisions 
produced by this accelerator and its very high luminosity, make possible the search
for physics BSM in a new kinematic regime. The data used for this dissertation comes from
pp collisions produced by the LHC at \sqrts 13 TeV, and recorded by the CMS experiment during 2016.
In the present Chapter, the LHC accelerator and the CMS experiment will be
described, making emphasis on the CMS subdetectors involved in the 
tau-lepton detection, which plays an important role in this work. 

%This Chapter is organized as follows: First, a brief description of the LHC accelerator is presented;
%next, the CMS subdetector and its subdetectors are described; and finally, the chapter finishes with a brief 
%description of physic objects reconstructed by CMS

\section{LHC Accelerator}
\label{sec:LHC}
The LHC is a particle accelerator designed to collide protons (or lead ions) at 
a \centermassenergy~up to 14 TeV. The LHC accelerates protons along 
two rings with a circumference of 27 km, installed in a tunnel 
approximately 100 m underground. Bunches of protons (or ions)
are accelerated in opposite directions using ratio frequency (RF) cavities along the rings. They 
collide in four different points, where dedicated experiments are placed 
in order to detect the products of the collisions. The four experiments
are: CMS \cite{chp2:CMSTDR,chp2:CMSTDR2}, 
ATLAS \cite{chp2:ATLASTDR}, LHCb \cite{chp2:LHCb} and  ALICE \cite{chp2:ALICETDR} (see Figure \ref{figchp2:LHCringsfigure}). CMS 
and ATLAS are multi-purpose detectors, optimized for the discovery of new physics BSM. The aim of the LHCb detector is to study 
the charge-parity (CP) symmetry violation, which has been postulated to explain the origin of matter-antimatter 
asymmetry in our universe. ALICE is specialized in studying the quark-gluon plasma. Besides these experiments, there 
are two additional smaller ones: TOTEM and LHCf. The TOTEM main goal is accurate measurement of total, elastic and diffractive pp
cross sections. LHCf uses the particles emitted forward by collisions in order to simulate
the cosmic rays behavior in controlled conditions. 

\begin{figure}[ht]
\begin{center}
 \scalebox{0.2}{
          %\includegraphics{figuras/Chapter2/LHCrings}}
          \includegraphics{figuras/Chapter2/LHCrings2}}
\caption{Schematic view of the LHC proton rings showing the interaction points
where the CMS, ATLAS, ALICE and LHCb detectors are located.}\label{figchp2:LHCringsfigure}
\end{center}
\end{figure}

\subsection{LHC Pronton Accelerator Chain}
\label{subsec:ProtonAcceleratorChain}

%at an energy of 450 GeV, they have several
In order to achieve the very high energy of the LHC proton beams, several
pre-acceleration stages are used as shown in Figure \ref{figchp2:LHCaccelerationchain}. Before 
protons are injected into the two LHC rings, the process starts 
with the extraction of protons in the \textit{Duoplasmatron 
Proton Ion Source}; in this source, the protons are extracted in ``\textit{bunches}'' 
by ionization of hydrogen gas. Then, these bunches are 
injected into a linear accelerator, \textit{Linac2}, where 
their energy is increased up to 50 MeV. Once the protons 
reach such energy, they are delivered to the 
\textit{Proton Synchrotron Booster} (PSB) and then to the \textit{Proton Synchrotron} (PS) 
where they are accelerated up to 1.4 GeV and 25 GeV, respectively. Additionally,
in the PS the bunches are spaced in time by 25 ns\footnote[1]{The time spacing between 
bunches is known as \textit{Bunch Crossing} (BX). For the LHC Run I, the BX was 50 ns; for Run II 
the BX was reduced to its design value of 25 ns.}. The pre-acceleration chain 
finishes in the \textit{Super Proton Synchrotron} (SPS), where the protons reach 
an energy up to 450 GeV, and then they are injected into the LHC rings. The bunches are 
injected into the LHC rings in opposite directions where they reach an 
energy up to 7 TeV per beam; i.e. the LHC could produce proton-proton collisions 
up to 14 TeV in the center-mass frame (\sqrts 14 TeV). The protons are accelerated 
using 8 RF cavities, which operate with a frequency 
that goes up to 400 MHz in the final acceleration stage and an electric field 
gradient of 5 MV per meter \cite{chp2:LHCTDR}. The RFs also ensure the longitudinal stability of the beams. The 
proton bunches are radially focused by 392 quadrupole magnets placed along the 
LHC rings. The radial focusing is important to increase the collision probability. Additionally, the 
LHC uses 1232 superconductor dipole magnets, where each of them produce a magnetic field strength 
up to 8.3 T, in order to bend the beam along the circular path. The dipoles 
are cooled by superfluid helium at a temperature of 1.9 K, that also helps to improve the vacuum
inside the beam pipes. 

\begin{figure}[ht]
\begin{center}
 \scalebox{0.35}{
          \includegraphics{figuras/Chapter2/LHCaccelerationchain}}
\caption{Schematic view of the proton accelerator chain at CERN.}\label{figchp2:LHCaccelerationchain}
\end{center}
\end{figure}


%since the protons within a bunch will be placed in a smaller cross-section. \\

\subsection{LHC Operational Parameters}
\label{sec:LHCparameters}

The main quantities that describe the performance of a particle accelerator are the beam energy and 
the luminosity. The instantaneous luminosity measures the number of collisions per unit of area 
and per unit of time \footnote[1]{The units of instantaneous luminosity are 
cm$^{-2}$s$^{-1}$.}; it depends on the accelerator's design and, in the 
case of the LHC, it is given by:

\begin{equation}
 L = \frac{N_{b}^{2}k_{b}f_{rev}\gamma}{4 \pi \epsilon \beta^{*}}R \;,
\end{equation}

\noindent where $N_{b}$ is the number of protons per bunch, $k_{b}$ is the number of bunches per beam, $f_{rev}$
is the number of revolutions per second, $\gamma$ is the relativistic factor, $\epsilon$ is the 
normalized transverse beam emittance, $\beta^{*}$ is the optical $\beta$-function at the
collision point and $R$ is a
geometrical factor related with the crossing angle of the two beams. The optimization of 
the instantaneous luminosity is achieved in several ways, such as by increasing the number
of protons per bunch ($N_{b}$) or by increasing the number of bunches per beam ($k_{b}$). For the statistical accuracy 
of the physics measurements, the important quantity is the integrated luminosity \footnote[1]{The units of the 
integrated luminosity are inverse barn, $b^{-1}$.} (instantaneous luminosity integrated in time, $\mathscr{L}$) because the number of events 
produced ($N_{eve}$) is proportional to it, and it is given by the expression:

\begin{equation}
 N_{eve} = \mathscr{L} \sigma_{p} \;,
\end{equation}

\noindent where $\sigma_{p}$ is the cross-section of the process of interest,
for example the cross-section of an hypothetical \Zprime~production, 
$\sigma(pp \rightarrow Z^{\prime})$. If the cross-section that
we want to measure is small, we need a large integrated luminosity
in order to achieve a reasonable number of events, even more, if the process
demands for a complex discrimination between signal and background. For instance, a 
\Zprime~decaying into two taus should have a cross-section of the order of pb, while 
Drell-Yan process, that constitute an important background, has a 
cross-section of around 5780 pb \footnote[7]{In this example, the cross-sections
are estimated for pp collisions at \sqrts 13 TeV and for a \Zprime~with a mass of 500 GeV.}.
Therefore, the signal discrimination in this case should be of the order of 10$^{-3}$,
demanding even higher luminosities. In addition, in each BX there are several pp
 collisions (of the order of 20, during 2016 data taking period), and only a very small 
 fraction of these collisions correspond to hard proton-proton interactions that could produce
 interesting physics. Most of the collisions correspond to elastic and diffractive scattering
 as well as soft scattering. This means that in a BX with an interesting hard 
 scattering collision there would be another 20 soft scattering ones, that will have to be 
 identified and remove from the event. This phenomenum is known as “pile-up” 
 (PU) and makes difficult the identification of the hard interaction. The level 
 of PU depends on the number of protons per bunch $N_{b}$, as well as on the geometrical 
 factors of the beam. \\
 
\noindent  In order to achieve the high statistics for the identification of BSM signals, a good performance 
 of the LHC accelerator is required. Operational quantities such as beam energy, luminosity, bunch crossing and etc., 
 have been improved from one running period to the next. Table \ref{chp2:LHCtable}
 shows the most relevant quantities of the LHC operation per year.
 
 
\begin{table}[ht]
\begin{center}
\begin{tabular}{|l|c|c|c|c|c|c|} \hline \hline
                                                     &                &                 \multicolumn{3}{|c|}{Run I}            &  \multicolumn{2}{|c|}{Run II}        \\    \cline{3-7}
                                                     &     Design     &       2010      &       2011       &       2012        &       2015       &       2016        \\     \hline \hline   
   Beam Energy [TeV]                                 &        7       &       3.5       &        3.5       &          4        &        6.5       &        6.5        \\ 
   Number of protons per bunch [10$\times^{11}$]     &      1.15      &       1.0       &        1.3       &        1.5        &        1.1       &        1.1        \\
   Number of bunches per beam                        &      2808      &       368       &       1380       &       1380        &       2244       &       2200        \\
   Bunch Spacing [ns]                                &        25      &       150       &         50       &         50        &         25       &        25         \\
   Average Pile-up in CMS                            &                &                 &                  &         21        &         14       &        27        \\
   Maximum peak luminosity [10$\times^{34}$ cm$^{-2}$s$^{-1}$] & 1.0  &       0.021     &        0.35      &         0.77      &        0.51      &       1.4        \\
   Integrated Luminosity [fb$^{-1}$]                 &                &       0.048     &         5.5      &         22.8      &         4.2      &       40.82       \\  \hline \hline
\end{tabular}
\end{center}
\caption{Some relevant operational parameters for Run II, compared with values reached during Run I \cite{chp2:LHCparameters}.}\label{chp2:LHCtable}
\end{table}

\subsubsection{Run I and Run II of the LHC}

The LHC operational period spanning from 2010 to 2012 is known as Run I. During 2010, the LHC 
reached pp collisions at \centermassenergy~of 7 TeV and an integrated luminosity of around 
50 pb$^{-1}$. During 2011 with the same \centermassenergy, an integrated luminosity of approximately 
6 fb$^{-1}$ was obtained. For the 2012 run, the energy per beam was increased 
up to 4 TeV (\sqrts 8 TeV), reaching approximately 20 fb$^{-1}$. The integrated luminosity delivered 
by the LHC and recorded by CMS during Run I is shown in Figure \ref{figchp2:luminosityRunI}. Afterwards, a two-years
technical shut-down took place. \\

\begin{figure}[ht]
    \begin{center}
      \includegraphics[width=0.47\textwidth]{figuras/Chapter2/CMSluminosityRunI.jpg}
      \caption{CMS integrated luminosity for proton-proton collisions delivered
      in Run I at the LHC. Taken from \cite{chp2:LHCluminosity}.}
     \label{figchp2:luminosityRunI}
    \end{center}
\end{figure}

\noindent After the shut-down, a new data taking period, known as Run II, started on May 
2015. That year, LHC delivered 4.2 fb$^{-1}$ of pp collisions at \sqrts~13 TeV. During 2016, LHC 
operated at the same \centermassenergy~than during 2015 and reached 40.82 fb$^{-1}$, exceeding  in 53$\%$ the expected integrated 
luminosity ($\sim$26 fb$^{-1}$). The integrated luminosity delivered by the LHC and recorded by CMS, 
during 2015 and 2016, is shown in Figure \ref{figchp2:luminosityRunII}. The success of the 2016 run was due to
the LHC operational stability and the high peak luminosity reached (1.4$\times10^{34}$cm$^{-2}$s$^{-1}$, which represented 
an improvement of 40$\%$ over the design value); this was achieved due to 
the shortening of the beam size from the injectors and the reduction of the 
crossing angle between the two beams. Compared with the 2015 run, the 2016 run 
reduced the number of protons per bunch to 1.1$\times10^{11}$ and the number of bunches per beam 
from 2244 to 2200, keeping the BX at 25 ns. As a result, the average
PU went from 14 in the 2015 run to 27 in the 2016 run (see \ref{chp2:LHCtable}). As already mentioned, 
the data used for this dissertation is the one recorded by CMS during 2016 run.
%During this period in order to ensure a good LHC performance for the expected high luminosity and a higher energy per beam. 
\begin{figure}[ht]
    \begin{center}
      \subfloat[2015]{\includegraphics[width=0.5\textwidth]{figuras/Chapter2/CMSluminosity2015.pdf}}
      \subfloat[2016]{\includegraphics[width=0.5\textwidth]{figuras/Chapter2/CMSluminosity2016.pdf}}
      \caption{CMS integrated luminosity for proton-proton collisions delivered
      in Run II at the LHC. Figures taken from \cite{chp2:LHCluminosity}.}
      %During 2016 LHC delivered 40.82 fb$^{-1}$ and CMS recorded 39.7 fb$^{-1}$. 
     \label{figchp2:luminosityRunII}
    \end{center}
\end{figure}

\section{The CMS Detector}
\label{sec:CMS}

The Compact Muon Solenoid (CMS) is, along with ATLAS, a multi-purpose detector designed 
with a broad physics program, which includes the understanding of the 
electroweak symmetry breaking through the Higgs mechanism, 
and the search for physics BSM, such as SUSY, extra dimensions, etc. The CMS 
detector is located 100 m underground in ``\textit{Point 5}'' of the 
LHC (see Figure \ref{figchp2:LHCringsfigure}). It is a hermetic detector around the collision
point, with a cylindrical shape that has a length of 21.6 m and a diameter of 14.6 m. One of 
the especial features of the detector is the superconducting solenoid which produces an inner magnetic field 
of $3.8$ T (over a volume of 341.7 m$^{3}$). The strong magnetic field bends the tracks 
of the charged particles coming from the interactions, with the purpose of identifying their 
electric charge and to accurately measure their momentum. Besides the solenoid, the detector 
is composed by four subsystems: encased inside the solenoid are the Tracker System, the 
Electromagnetic Calorimeter (ECAL) and the Hadron Calorimeter (HCAL); and outside the 
magnet are the Muon Chambers. The purpose of the Silicon Tracker is to reconstruct 
the collision vertices and the tracks of the charged particles emerging from the 
collision. The Calorimeter system (ECAL and HCAL) allows to measure the energy of hadrons, electrons and 
photons. The Muon Chambers, embedded inside an iron-yoke structure, reconstruct the 
muon tracks and provide an accurate information for their momentum measurement. The overall layout of CMS detector is shown 
in Figure \ref{figchp2:CMSdetectorfigure}. A more detailed description can be found in Ref.~\cite{chp2:CMS}. \\

\begin{figure}[ht]
    \begin{center}
      \includegraphics[width=0.7\textwidth]{figuras/Chapter2/CMSdetector.pdf}
       \caption{CMS Detector overview \cite{chp2:CMSTDR}.}\label{figchp2:CMSdetectorfigure}
\end{center}
\end{figure}

\noindent All the information produced in an event is stored in the readout electronics 
of each sub-detector. Once the event information is compressed, the data size
for one bunch crossing is around 1 MB therefore, with the nominal LHC luminosity, CMS would 
produce 40 TB of information per second; this high rate makes impossible 
the data storage with the current technology. Nevertheless, interesting physics can be produced only 
in hard-interaction collisions, which occur approximately once each one million 
collisions. In consequence, CMS uses a Trigger System to select only the hard 
scattering events which are reduced from 40 million collisions per second 
to only 100 events per second, making feasible the data storage. Once the 
Trigger System identifies the interesting events, they are stored in order to be 
analyzed afterward. \\

%PARRAFO HAY QUE MEJORAR TRIGGER Y DAQ Y VOLUMEN DEL SOLENOIDE
%The high collision rate that the LHC produces makes the data storage impossible 
%with the current technology. In consequence, CMS uses a Trigger System to select 
%only the hard scattering events, which are the most likely to have some interesting 
%physics. From 40 million collision per second produced by the LHC, the Trigger 
%System selects and stores only 100 events per second. \\

\textbf{Coordinate System}\\

%CMS is divided into divided into a barrel (central region) and two endcaps.
%In order to locate in space not only the detector components but also the reconstructed
%particles CMS has defined a coordinate system. 

\noindent In order to define the position of any detector components and, in consequence,
of any particle signal, CMS has defined a cartesian and spherical 
coordinate systems. The origin of both systems is the nominal interaction point, which is at the center
of the detector. The x-axis points towards the center of the LHC ring,
the y-axis points upwards and the z-axis points along the beam pipe in the
counterclockwise direction. The polar angle $\theta$ and azimuthal angle $\phi$ are defined 
in the usual way. \\

\noindent In order to parametrize the direction in which the particles are emitted,
the pseudorapidity $\eta$ is better than $\theta$ because its distribution
is more uniform. Pseudorapidity is given by:

\begin{equation}
 \eta = -ln \left( tan \left(\frac{\theta}{2} \right) \right)
\end{equation}

%characterize the is measured from the x-axis in the transverse plane to the beam (x-y plane), while the 
%polar angle $\theta$ is measured from z-axis. %, the polar coordinate used is The pseudorapidity $\eta$ of the particles is related to the polar angle
%instead of $\theta$ %because the distribution of particles is more uniform for $\eta$ than $\theta$. 

\noindent A more detailed description of the CMS coordinate system can be 
found in Ref.~\cite{chp2:CMS}. 

\section{Superconducting Solenoid}
\label{sec:Solenoid}

\noindent The superconducting solenoid produces a uniform inner magnetic 
field with the value of 3.8 T. In the outer region, the 
returning magnetic flux is compactified by the iron yoke, resulting 
on an average magnetic field strength of 2 T. The magnetic field 
provided by the solenoid bends the tracks of the charged particles 
emerging from the collisions (see Figure \ref{figchp2:CMStrajectories}) which is crucial
for their charge identification and their momentum measurement. The electric charge 
is identified according to the direction of the
bending in the magnetic field. The momentum of a charged particle 
that moves through a uniform magnetic field is given by:

\begin{equation}
p = \gamma m v = qBr \;, 
\end{equation}

\noindent where $\gamma$ is the relativistic factor; $m$, $v$, $q$ are its mass,
rapidity and charge, respectively; $B$ is the magnetic field strength; and $r$ is the ratio 
of the bending. A strong magnetic field is necessary for the momentum 
measurement of very energetic charged particles, in consequence the momentum resolution 
depends on the magnetic field strength and the spatial resolution of the detectors 
that reconstruct the tracks. The momentum resolution is:

\begin{equation}
 \frac{\sigma_{p}}{p} \propto \frac{p}{BL^{2}} \;,
\end{equation}

\noindent where $L$ is the length of the trajectory. The strong magnetic field provided by the solenoid 
and the high spatial resolution of the tracker detectors, allows CMS to have 
a very good momentum resolution; for instance, the momentum resolution for muons 
is 1$\%$ up to 100 GeV \cite{chp2:CMSTDR2}.\\

%Figure  shows the 
%trajectories for several particles in the solenoid magnetic field.

\begin{figure}[ht]
    \begin{center}
      \includegraphics[width=0.7\textwidth]{figuras/Chapter2/CMStrajectories.png}
      \caption{Schematic view on CMS transverse plane for trajectories and energy deposits of several 
      particles moving through the solenoid magnetic field. Taken from \cite{Barney:2120661}.
      } \label{figchp2:CMStrajectories}
    \end{center}
 \end{figure}

\noindent The CMS superconducting solenoid is the biggest one built ever. The coil is made of NbTi with 4 layers of winding. In order to 
generate the strong magnetic field, this solenoid operates
with a nominal current of 19.5 kA, storing an energy up to 2.6 GJ. A cooling system
keeps its superconducting state using liquid helium at 4.65 K. The 
main parameters of the CMS magnet are summarized in Table \ref{tablechp2:Solenoid}.

\begin{table}[h]
\centering
\begin{tabular}{|l|c|}\hline \hline
\multicolumn{2}{|c|}{Magnet Parameters}  \\ \hline \hline
Inner magnetic field   &  3.8 T  \\ \hline
Diameter               &  5.9 m  \\ \hline
Length                 &  12.5 m  \\ \hline
Nominal Current        &  19.5 kA \\ \hline
Stored Energy          &  2.7 GJ  \\ \hline
Inductance             &  14.2 H  \\ \hline \hline
\end{tabular}
\caption{Main parameters of the CMS Solenoid.} \label{tablechp2:Solenoid}
\end{table}



% Weighting 10000 tonnes, this magnet is 12.5 m long and 6 m diameter. 


\section{The Tracker System}
\label{sec:Tracker}

\noindent The Tracker System is the inner-most detector system in CMS. It has 
a cylindrical shape covering an acceptance range of $|\eta| < $ 2.5. This system is 
designed to reconstruct accurately the charged-particle tracks, which is essential 
for achieving a high momentum resolution as well as for identifying the primary and the 
secondary vertices. Secondary vertices are the evidence of the decay 
of long-lived particles coming out the collision, for instance 
jets originated from b-decays, known as \textit{b-jets}, which have a vertex 
that can be distinguished from the primary vertex. Therefore 
a high resolution of secondary vertex reconstruction is required in order to 
identify b-jets (see section \ref{sec:bJet}). In this dissertation the 
Tracker System is crucial due to the following reasons:

%for the 
%The Tracker System, 
%besides the identification of primary vertex and the reconstruction of 
%charged-hadron tracks, is important in this dissertation for several reasons:

\begin{itemize}
 \item The searches for heavy resonances in the di-lepton channels require a 
 high momentum resolution for leptons with transverse momentum greater than 1 TeV. 
 \item Since the tau-leptons can decay into charged hadrons (see section \ref{sec:Taus}), the 
 reconstruction of these hadrons in the Tracker System is crucial
 for the tau identification and reconstruction. Charged
 hadrons are reconstructed with an efficiency of at least 95$\%$ for \pt~greater than
 10 GeV (see section \ref{subsec:TrackReco})\cite{TrackerPerformace}.
 \item The transverse impact parameter and secondary-vertex resolution are comparable with 
 the distances that tau-leptons travel before decay \cite{TauReconstructionCMSRun1}; 
 in consequence, there is a probability that a b-jet can fake 
 the tau signature \textcolor{red}{REFERENCIA A LA SECCION DONDE SE HABLA DE B-JET BACKGROUND}. The 
 high b-jet identification efficiency provided by the Tracker System will reduce this source of background.
 \item Due to the material of the Tracker System, there is a high probability 
 that photons coming from $\pi^{0} \rightarrow \gamma\gamma$ convert into electron-positron 
 pairs. The interaction of these photons with the Tracker material is important for jet and tau reconstruction,
 since $\pi^{0}$'s are copiously produced in QCD jets and they are also one of the tau decay 
 products (see section \ref{sec:Taus}).
 %performance of their reconstruction in the Tracker System 
%Since tau-lepton decays on charged hadrons (see section \ref{sec:Taus}), 
 %The tau reconstruction performance and its momentum resolution depend on 
 
 % which is important for the tau reconstruction since, as 
% mentioned in section \ref{sec:Taus}, $\pi^{0}$ is one of the tau decay products.
\end{itemize}

\noindent Since the Tracker System is the closest detector to the interaction point, it is exposed 
to the highest radiation doses in CMS; in average 1000 particles from 27 proton-proton collision 
are passing through this system each 25 ns. In consequence, the Tracker System was designed to achieve a high granularity 
and a fast response in order to identify the tracks and associate them 
to the proper BX. Additionally, this system was designed to have a high level of radiation hardness. Since 
these requirements are fulfilled by silicon detectors, the Tracker System is based on 
this technology. This system is composed of two subsystems: the Pixel Tracker and the Strip Tracker. \\ 

%\subsection{Pixel Tracker}
%\label{subsec:Pixel}

\textbf{The Pixel Tracker}\\

\noindent The Pixel Tracker is the closest subsystem to the beam pipe. It is 
composed by three concentric layers with a radius of 4.4 cm, 7.3 cm and 10.2 cm, and each one has a lenght of 53 cm. In 
the forward regions, there are two disks in each endcap which are located at $|z|$ = 34.5 cm and $|z|$ =  46,5 cm from 
the interaction point (see Figure \ref{figchp2:Tracker}). The whole subsystem has 
approximately 66 million pixels, covering an active surface area of around
one squared meter. Each pixel cell has a size of 100$\times$150 $\mu$m$^{2}$, providing a high 
spatial resolution in the $r-\phi$ plane and also in the $z$ direction. The Pixel detector has 
an acceptance range of $|\eta| < 2.5$ \cite{chp2:CMS}.\\

%\textcolor{red}{REFERECIA PIXEL}
%In summary, the Pixel detector provides three accurate points in an acceptance range of $|\eta| < 2.5$, mainly 
%used for vertices reconstruction. \\

\textbf{The Strip Tracker} \\

\noindent Surrounding the Pixel Tracker is the Strip Tracker, which is made of 9.6 million strips sensors,
covering an active detection area of 198 m$^{2}$. This subsystem is divided in two 
components: the inner tracker (20 cm $< |z| <$ 55 cm) and the outer 
tracker (55 cm $< |z| <$ 116 cm). The inner tracker
is made of 4 cylindrical layers, called \textit{Tracker Inner Barrel} (TIB), and 3 disks
installed in each endcap, known as \textit{Tracker Inner Disks} (TID). The outer tracker
is composed of 6 layers, called \textit{Tracker Outer Barrel} (TOB), and 9 disks in each 
endcap, known as \textit{Tracker EndCaps} (TEC). The whole Strip Tracker has 
a diameter of 2.4 m and a length of 5.5 m, covering an acceptance range 
of $|\eta| < 2.5$ (see Figure \ref{figchp2:Tracker}). The spatial resolution depends 
on the location of each strip sensor within the subsystem, for instance, the spatial 
resolution provided by the TIB which varies from 23 to 34 $\mu$m in the $r-\phi$ plane and 23 $\mu$m in 
the $z$ direction. \\

\begin{figure}[ht]
    \begin{center}
      \includegraphics[width=0.75\textwidth]{figuras/Chapter2/Tracker.png}
      \caption{Layout of the Tracker System detectors. The blue part represents 
      the Pixel Detector and the red part represents the Strip Tracker: TIB,
      TID, TOB and TEC. Taken from \cite{chp2:CMS}.} \label{figchp2:Tracker}
    \end{center}
 \end{figure}

\textbf{Particle interaction with the Tracker material} \\

\noindent When a particle passes through the Tracker System it interacts not only with the active volume
of the detector, but also with the other components such as the read-out electronic, 
the mechanical structure, the services and the cooling system. This amount material is significant
and the interaction with the crossing particles must be considered. Figure \ref{figchp2:Tracker_TrackerMaterial}
shows the thickness (in terms of number of radiation lengths) of tracker material that a particle must 
pass through before reaching the ECAL. As a result, there is a high probability that 
photons, coming from $\pi^{0} \rightarrow \gamma\gamma$, can convert into electron-positron 
pairs which, as mentioned above, is important for jet and tau reconstruction.

\begin{figure}[ht]
    \begin{center}
      \includegraphics[width=0.45\textwidth]{figuras/Chapter2/Tracker_Thickness.png}
      \caption{The total Tracker material thickness (t) in units of radiation length X$_{0}$, as a 
      function of $\eta$. Taken from \cite{TrackerPerformace}.
      } \label{figchp2:Tracker_TrackerMaterial}
    \end{center}
 \end{figure}

\section{The Calorimeter System}

\noindent The purpose of the Calorimeter System is to ``stop'' most of the particles 
coming out from the collision and to measure their energy with a good 
granularity. Electrons, photons, and hadrons interact with the calorimeter 
material and deposit all their energy on the detectors, allowing their energy 
to be measured. The only SM particles that scape from the Calorimeter System are 
muons and neutrinos: muons deposit a very low amount of their energy on the 
calorimeters, and therefore there are additional detectors installed in the outer side 
of CMS in order to identify them. In case of neutrinos, they escape without detection
since they only interact weakly, which leads an energy imbalance in the event; therefore 
the energy imbalance in an event is an indirect evidence of particles that only interact weakly, such 
as neutrinos \footnote[1]{A big amount of energy 
imbalance would be an evidence of dark matter candidates, since they only participate on weakly-interactions.}. \\

\noindent The CMS Calorimeter is divided in two subsystems: the electromagnetic calorimeter (ECAL) and 
the hadronic calorimeter (HCAL). The ECAL is the closest calorimeter subsystem to the interaction point, 
it is designed to measure the energy of electrons and photons. The reconstruction of photons and 
electrons are essential for many physics analyses, for instance for those analyses which 
jets or taus are involved; for these analyses, the photon and electron 
reconstruction are crucial for the energy measurement of the jet and for 
the tau identification. The HCAL, the other calorimeter subsystem, is designed to measure the energy of the 
hadrons. Figure \ref{figchp2:CMStrajectories} shows the arrangement of the 
Calorimeter System and the energy deposits of the SM particles on it.


%a jet contains a copious amount of 
%neutral pions which most of the times decays into one photon pair; since
%these photons might convert into electron-positron, the electron and 
%photon reconstruction is important for the energy measurement of the jet. Similarly, their
%reconstruction is important for tau identification since neutral pions are one of the decay 
%products of the tau-lepton. 

%because photons, that come from $\pi^{0}$ 
%decays ($\pi^{0}\rightarrow\gamma\gamma$), can convert into electron-positron pair; therefore, their 
%reconstruction is important for jet and tau identification, considering that $\pi^{0}$'s are produced copiously in 
%QCD jets and they are one of the tau decay products. 

%the granularity of the calorimeter is good enough to associate 
%the energy deposits of a charged particle with its reconstructed track on the 
%Tracker System

\subsection{The Electromagnetic Calorimeter}
\label{subsec:ECal}

\noindent The Electromagnetic Calorimeter is designed to absorb the total energy 
of electrons, and photons when they pass through it. The interaction between these 
particles and the calorimeter material produces a cascade of electrons, and photons in a process 
called electromagnetic showers. An electromagnetic shower starts when an energetic electron, 
or photon enters to the high density material of the calorimeter and starts to lose energy
due to its interaction with the calorimeter material. Electrons and 
positrons losses their energy by bremsstrahlung radiation, while photons 
losses their energy by electron-positron conversions. Therefore, a cascade
of secondary particles is created and the energy losses continue through 
these two processes (pair production and bremsstrahlung) until the energy of photons is not
enough for pair production and the energy losses of electrons are dominated by other processes 
than bremsstrahlung. As a result, the energy deposits of the electromagnetic shower 
are spread on the calorimeter material; then, the energy of 
the products of the electromagnetic cascade is measured using scintillators. The
the energy profile (transverse and longitudinal) of the electromagnetic shower is 
required to determinate the initial energy of the particle. Electromagnetic showers 
are then described by two parameters which depend
on the calorimeter material: the radiation length (X$_{0}$) and the Moli\`ere radius. The radiation length is 
the distance than an electron or photon travels until its energy is reduced by a factor 
of $1/e$, while the Moli\`ere radius is the radius of a cylinder where 90$\%$ of 
the electromagnetic shower is contained. In consequence, the calorimeter 
is designed with a enough thickness (in terms of radiation length) to measure the total
energy of the particles, and with a Moli\`ere radius small enough to achieve a good granularity. 
The schematic view of an electromagnetic shower is shown in Figure \ref{figchp2:EMshower}. 

\begin{figure}[ht]
    \begin{center}
      \includegraphics[width=0.3\textwidth]{figuras/Chapter2/ElectroCascade.png}
      \caption{Schematic view of the development of an electromagnetic shower. Taken from \cite{EMFigure}.
      } \label{figchp2:EMshower}
    \end{center}
 \end{figure}

\noindent The ECAL is a hermetic and cylindrical calorimeter whose scintillators are made of 
lead tungstate crystals (PbWO$_{4}$), see Figure \ref{figchp2:ECALcrytal}. The PbWO$_{4}$ crystals are
characterized by its high density (8.28 $g/cm^{3}$), its short radiation length (0.89 cm) and its
small Moli\`ere radius (2.2 cm), making them appropriate to achieve 
accurate energy measurements with a good granularity \cite{chp2:CMS}; additionally,
these crystals are radiation hardness and have a fast response, which make them
suitable for the LHC environment. When these crystals are crossed by electrons 
or photons, they emit a light pulse with a time response of $\sim$25 ns (about 80$\%$ of the times); that scintillator
light is collected by photo-detectors that convert it into an electric signal. The photo-detectors used in the barrel 
are Avalanche Photo-Diodes (APDs) while in the endcaps Vacuum Photo-Triodes (VPT) are used. \\

\begin{figure}[ht]
    \begin{center}
      \includegraphics[width=0.33\textwidth]{figuras/Chapter2/ECALcrytal.jpg}
      \caption{CMS ECAL Lead Tungstate Crystals (PbWO$_{4}$).
      } \label{figchp2:ECALcrytal}
    \end{center}
 \end{figure}
 
\noindent The ECAL has 61200 crystals in the barrel and 7324 crystals in each endcap \cite{chp2:CMSTDR}, covering a 
$|\eta|$ range up to 3. The crystals are installed in a quasi-projective geometry 
in such a way that each crystal points toward the center of the detector with an additional angle 
of 3$^{\circ}$ with respect to this direction \footnote[1]{In the case of the endcaps, this angle 
varies from 2$^{\circ}$  to 8$^{\circ}$.}; the purpose of this 
geometrical arrangement is to avoid particles passing through inactive regions of the ECAL. In the 
barrel region, the calorimeter (ECAL Barrel or EB) has a inner radius of 129 cm, covering a 
pseudorapidity region of $|\eta| < $1.479. The EB crystals are 230 mm long (25.8 $X_{0}$) and 
covers a cross-section at the front face of 0.0174 $\times$ 0.0174 
in the $\eta-\phi$ plane ($22\times22$ mm$^{2}$). In the forward region,
the ECAL (ECAL Endcaps or EE) is located at 314 cm from the collision point, covering 
1,479 $ < |\eta| < $ 3,0. The EE crystals are 220 mm long (24.7 $X_{0}$) and cover
a cross-section of 28.62 $\times$ 28.62 mm$^{2}$. In addition to the crystals, Preshower 
Detectors are installed in the endcaps, covering a pseudorapidity range of  $1.65<|\eta|<2.6$. This detector consists of two 
lead layers, each one followed by silicon sensors. Their main goal is 
to identify the two photons produced by neutral-pion decays 
in the forward region. The whole ECAL arrangement is shown in Figure \ref{figchp2:ECALcrytal}

\begin{center}
\begin{figure}[h]
\centering
\includegraphics[scale=0.38]{figuras/Chapter2/ECAL.pdf}
\caption{Arrangement of the ECAL components.}\label{figchp2:ECALcrytal}
\end{figure}
\end{center}

\noindent The energy resolution reached by the ECAL is given by:
\begin{equation} \label{ECALenergyResolution}
 \left(\frac{\sigma_{E}}{E}\right)^{2} = \left(\frac{2.8\%}{\sqrt{E}}\right)^{2} + \left(\frac{0.12\%}{E}\right)^{2} + \left(0.3\% \right)^{2} \;, 
\end{equation}

\noindent where 2.8$\%$ relies on the stochastic term due to electromagnetic showers, 12$\%$ comes from the 
electronic noise and 0.3$\%$ refers to the systematic uncertainty produced by the calibration of the apparatus. As can be inferred from
Eq. \ref{ECALenergyResolution}, the energy resolution for energetic particles is dominated by the systematic uncertainty, while
the stochastic and noise terms are dominant for low energies. As an example, the energy resolution 
obtained in the EB is close to 1$\%$ for all electrons that come from Z decays \cite{ECALperformance}.

\subsection{The Hadronic Calorimeter}
\label{subsec:HCal}

\noindent The Hadron Calorimeter (HCAL) is designed to stop the hadrons and to measure 
their energy. It is composed of fluorescent scintillators inserted in layers of 
a dense material called the absorber. The absorber is made of stainless 
steel and copper layers. When a hadron hits the absorber it produces a cascade of 
particles, known as hadronic shower. The hadronic shower is a cascade of 
secondary particles, mainly pions, originated by strong interactions between 
the hadron and the atomic nuclei of the absorber material.  Due to the copious pion
production, electromagnetic showers are also involved because of the 
neutral pion decays. As the hadronic shower develops, the particles are detected by the scintillators, which
 produces light pulses when a particle passes through it; then, the light pulses are 
 collected by optical fibers. The amount of light collected is proportional to the amount 
 of energy deposited in the scintillator material, allowing the energy to be measured. A hadronic 
 shower is described by the absorption length, which is the average distance traveled by the 
 hadron trough the medium before it strongly-interacts with the absorber; it is given by:

 \begin{equation}
 \lambda = \frac{A}{N_{A} \sigma_{abs}}  \;,
\end{equation}

\noindent where $A$ is the nuclei weight, $N_{A}$ is the Avogadro's number and $\sigma_{abs}$
is the absorption cross-section.\\

\noindent The HCAL arranged in the central pseudorapidity region, or 
HCAL Barrel (HB), is radially restricted since it is placed within the remaining volume
between the ECAL and the solenoid; for this reason, the Hadron Outer Calorimeter (HO) is 
installed outside the solenoid in order to improve the hermiticity. In the 
forward region, there are two subsystems installed: the HCAL Endcap (HE) 
which covers the pseudorapidity range acceptance of $\eta <$ 3, and the HCAL 
Forward (HF) which extends the $\eta$ coverage of the calorimeter up to 5.2. The HCAL structure 
is shown in Figure \ref{figchp2:HCAL}.

%3$ < |\eta| < $5.2.


\begin{center}
\begin{figure}[h]
\centering
\includegraphics[scale=0.4]{figuras/Chapter2/HCAL1}
\caption{Disposition of the HCAL components.}\label{figchp2:HCAL}
\end{figure}
\end{center}

\noindent The HB is divided into two identical halves in the $z$-direction. Each half is composed of 18
wedges, covering a range of $|\eta| < $ 1.4. Each wedge consists of brass absorber plates 
staggered with plastic scintillators. The absorber material is made of brass 
because it is a non-magnetic material and it has a small 
radiation length ($\lambda = $ 16.42 cm), making it appropriate for the CMS environment. Between every
two layers of absorbers, there is installed a plastic scintillator with a thickness of 3.7 mm; the scintillators
send the light pulse through optical-fibers (WaveLenght Shifter, WLS) to reach the 
photo-detectors (Hibryd Photodiode, HPD). The photo-detectors convert the light pulse 
into the electric signal. The innermost and 
outermost plates of the wedge are made of stainless steel in order to strengthen the structure. Each wedge 
is segmented in the $\eta$-direction by 16 structures called towers; each tower 
covers an identical solid angle whose area in the $\Delta\phi\times\Delta\eta$ plane is 0,087$\times$0,087. \\

\noindent The HO is located outside the solenoid, covering $|\eta| < $ 1.26. It uses as absorber 
material the solenoid itself and the most inner layer of the 
iron-yoke structure. The HO is divided into
5 wheels in the $z$-direction, similar to the geometrical disposition 
of the Muon Chambers (see section \ref{sec:MuonSys}). In the central wheel of the detector, there 
are two layers of plastic scintillators staggered with the solenoid and the inner 
layer of the iron yoke. Additionally, there is one scintillator installed in 
each remaining wheel just behind the solenoid. The Figure \ref{figchp2:HCAL} shows the disposition of 
the HO plastic scintillators. \\

\noindent In the forward region, the HE is installed just behind the EE, covering 
1.4 $ < |\eta| <$ 3. The HE is a sampling detector, similarly than the HB, which 
is composed of brass layers staggered with plastic scintillators of 3.7 mm of 
thickness. The HF is located at 11.2 m from the interacting point in the $z$-direction,
covering an extended $\eta$ region up to 5. The purpose of 
the HF is to achieve a better hermiticity, which improves 
the reconstruction of very forward jets as well as 
the measurement of the energy imbalance in an event. \\

\noindent Due to the nature of a sampling detector as the HCAL, its energy resolution 
is worse than the one obtained by the ECAL \cite{chp2:CMS}. For example, the energy
resolution for pions is given by:

\begin{equation} \label{ECALenergyResolution}
 \left(\frac{\sigma_{E}}{E}\right)^{2} = \left(\frac{138\%}{\sqrt{E}}\right)^{2} + \left(13\% \right)^{2} \;, 
\end{equation}

\noindent Although the HCAL does not provide a high energy resolution, it is not crucial
since the energy resolution is improved combining the information obtained by the HCAL
with the other detector systems. 

\section{The Muon System}
\label{sec:MuonSys}

\noindent The purpose of the Muon System, additional to the muon identification and reconstruction, is
to trigger with muons since their presence in an event might indicate interesting physics. The Muon System
is located outside the solenoid since the muons are the only known charged particles 
than can cross the tracker, the calorimeters and the solenoid material. In order to achieve the
muon reconstruction and to trigger with them, CMS has three different kinds of 
Muon Detectors: Drift Tubes (DTs), Cathode Strip Chambers (CSCs) and Resistive 
Plate Chambers (RPCs). The DTs and the CSCs are used for tracking the muon 
signature and for measuring its momentum; they are placed in the barrel and 
the endcaps, respectively. The RPCs are used for triggering due to their high time 
resolution; they are used in both the barrel and the endcaps. The Muon System structure is 
shown in Figure \ref{figchp2:MuonSystem}. In summary the three technologies used in the Muon Chambers are:

\begin{itemize} 
 \item \textbf{Drift Tubes}, or DTs, are gaseous detectors used for tracking due to their excellent spatial resolution.
The DT basic cell is a tube filled with a gas mixture of 85$\%$ Ar and 15$\%$ CO$_{2}$.
The anode is an aluminum wire placed longitudinally in the center of the tube.
The tube has a cross section of 13 $\times$ 22 mm$^{2}$ and a length between 2-3 m depending on
to the position of the chamber in the barrel. When a charged particle passes through the cell 
the gas is ionized and produces an avalanche of electrons because of the high 
electric field close to the wire. The wire picks up the electrons and the signal is later amplified. The
accurate track reconstruction is performed by the structure of interleaved consecutive DT layers. 
 \item \textbf{Cathode Strip Chambers}, or CSCs, performs the muon track reconstruction in the endcaps. Each CSC chamber
has a trapezoidal shape in order to cover the 12 azimutal sectors of the detector. A CSC Chamber 
has 6 planes of wires along the azimuthal direction, defining the radial coordinate of the track. The wire
planes are intercalated with 7 panels of cathode strips that run radially and give the $\phi$ measurement 
with a resolution of 10 mrad. There are a total 
of 540 CSCs, where 72 new Chambers installed in the forth disk during the LS1. 
\item \textbf{Resistive Plate Chambers}, or RPCs, are gaseous parallel-plate detectors with a time resolution of about 1 ns.
This detector has an important roll for the trigger system since their excellent time resolution is suitable to
identify unambiguously the BX associated to the muon detected. There are  6 RPC layers in the barrel
and 4 disks in each endcap; the fourth disk was installed during LS1 (see Figure \ref{figchp2:MuonSystem}).
 \end{itemize}


\begin{center}
\begin{figure}[h]
\centering
\includegraphics[scale=0.4]{figuras/Chapter2/MuonSystem}
\caption{Longitudinal view of CMS, showing the Muon System layout: DT Detectors are represented in pink,
CSCs are green and RPCs are blue. GEM Detectors, that will be installed in LS2, are in red.}\label{figchp2:MuonSystem}
\end{figure}
\end{center}

\noindent In the barrel region, the Muon Chambers (DTs and RPCs) are inserted into the 
five wheels of the iron yoke; they form four concentric cylindrical 
layers, each layer is divided in 12 sectors in $\phi$. Then, the geometrical 
arrangement of the Muon System consists on 5 wheels in the $z$-direction, 4 layers in 
the $r-$direction and 12 sectors in the $\phi-$direction. In the forward region, the Muon Chambers (CSCs and RPCs)
are arranged in each endcap into 4 disks, where each disk is divided into 
12 sectors in the $\phi-$direction and 4 rings in the $r-$direction. \\

\noindent The geometrical disposition of the Muon Chambers, the DTs and the RPCs in the barrel
and the CSCs and the RPCs in the endcap, composes a complementary system which provides
two independent sources of information in each region, which improves the
muon reconstruction efficiency. The RPCs provides a fast response ($<$ 25 ns), allowing 
the muon reconstructed track to be associated to a BX unambiguously. The DTs and CSCs provide 
a high spatial resolution, providing an accurate energy 
and momentum measurements. The information obtained from the Muon Chambers
are combined with the one provided by the Tracker system in order to 
improve the muon identification efficiency and the muon reconstruction. For instance,
the momentum resolution for muons of \pt~$= 10$ GeV is 10$\%$ using only the information
provided by Muon Chambers, while it decreases up to 1$\%$ combining their information with 
the one obtained from the Tracker System \cite{chp2:CMSTDR}.\\

\noindent During the Long Shot Down, or LS2, Gas Electron Multipliers (GEMs) will be installed in 
the endcaps in order to improve the muon identification efficiency and
the momentum measurement resolution in these regions (see Figure \ref{figchp2:MuonSystem}). In 
addition to their high granularity, GEMs will be used for triggering,
due to their high time resolution.

\section{The Trigger and Data Acquisition Systems}
\label{sec:Trigger}

\noindent The Trigger and the Data Acquisition (DAQ) Systems use the information 
provided by the Tracker, the Calorimeter, and the Muon Systems in order 
to identify the events with interesting physics and to store them to be 
analyzed afterward. The information of an event is stored for a few $\mu s$ in 
the read-out electronics of each sub-detector system; when these pieces of 
information are combined, the data size of an event is around 1 MB, which means 
that for each bunch crossing, 1 MB of data is produced. Considering that the LHC 
operates with a collision rate of 40 MHz at nominal luminosity, CMS could 
deliver up to 40 TB of information per second; although the data storage 
for such high rate is impossible with the current technology, only the hard
scattering collisions (interesting events) should be selected and stored. In 
consequence, CMS uses the Trigger System to select only the interesting events, which 
reduces the event rate from 40 MHz to 100 Hz, making feasible the data 
storage. Additionally, the data storage in the read-out electronics of 
each detector has a limited time before the information is erased; therefore, the 
read-out electronics keep the event information stored
temporarily until it receives a signal from the Trigger System 
to store the event definitely; but while the Trigger System is taking the decision 
if an event is stored or discarded, the read-out electronics keep storing the 
information of the following collisions. Therefore the data storage must 
be synchronized with the Trigger System. Due to its complexity, CMS uses the 
Data Acquisition System (DAQ) in order to storage the data.

\subsection{The Trigger System}
\label{subsec:Trigger}

\noindent The Trigger System performs the selection of the hard scattering events in two steps:
The Level-1 Trigger (L1 Trigger) and the High Level Trigger (HLT). The first step, the Level-1 Trigger,
is performed by programmable read-out electronics in order to reduce the event rate from 40 MHz 
to 100 kHz. The second step, the HLT, is performed by about one thousand processors, which through
software algorithms reduces the event rate up to 100 Hz. \\

\textbf{The Level-1 Trigger}\\

\noindent The Level-1 Trigger performs the first event selection through hardware criteria programmed 
in the read-out electronics of each sub-detector. A hard scattering event typically 
involves heavy particle production such as b-quarks, t-quarks, heavy bosons 
or hight-\pt~muons; then, interesting physics events usually content jets and muon signals. In 
consequence, the Level-1 Trigger searches for signals into the read-out electronics 
of the Calorimeter System and the Muon System. The information provided by these two sub-detectors
is combined to be sent further to the Global Trigger System; this system takes the decision
if the event is accepted sending a signal, known as L1-Accept, to the read-out electronics 
in order to store the information \footnote[1]{If the L1-Accept signal is not produced for an event, it is erased from the 
readout electronics memories.}. Once the event is stored by the DAQ System, the data is moved 
to the HLT. Figure \ref{figchp2:TriggerSystem} shows the data flux performed by the L1 Trigger System
in order to produce the L1-Accept signal. \\

\begin{center}
\begin{figure}[h]
\centering
\includegraphics[scale=0.35]{figuras/Chapter2/trigger}
\caption{The Level-1 Trigger architecture. The L1-Accept signal is triggered when 
the event passes the three components: local, regional and global.}\label{figchp2:TriggerSystem}
\end{figure}
\end{center}

\textbf{The High Level Trigger}\\

\noindent As mentioned above, the selected events by the L1 Trigger are delivered 
to a farm of about one thousand processors, where the High Level Trigger is implemented. 
Unlike L1 Trigger which event selection is hardware based, the HLT selection is 
software based where a more complex requirements are applied. The event selection by the HLT starts 
with the readout of the information that comes from each sub-detector system;
then, the processors implement complex algorithms in order to reconstruct
and to select events with interesting physics; the selected data is then moved
to on-line servers with the purpose of studying the CMS performance, in a process 
known as Data Quality Monitoring or DQM. Finally, those events that pass the HLT trigger (100 events per second) 
are stored in disk. When the data is stored, they are labeled and grouped 
according to the the physics objects identified in each selected event; for instance, 
all the events triggered due to the presence of a muon are grouped into a dataset, hence 
an research which involves muons in the final state can use the muon dataset for analysis. 


%the collected data is moved to
%on-line servers with the purpose of studying the CMS performance, in a process known as Data Quality
%Monitoring or DQM. 




\subsection{Data Acquisition System}
\label{subsec:Trigger}

\noindent The Data Acquisition System (DAQ) works synchronously with the Trigger System
in order to store the hard-interaction events. The information of an event 
is stored in read-out electronic devices in pipeline memories meanwhile 
the L1 Trigger decision is being taken; then if the read-out electronics
receive the L1-Accept signal, the event will be delivered to the HLT System, otherwise
the event will be erased. Therefore, DAQ system is designed to operate with 
same input frequency than the L1 Trigger and delivers the data with a similar 
rate than the HLT. For this purpose electronic devices known as 
front-end systems (FES) store data continuously in 40-MHz pipelined 
buffers, where the information of an event is stored 
for 3.2 $\mu s$, time in which the L1-Accept signal can reach FES electronics. When 
L1 Accept signal arrives to the FES, the information is pushed into the DAQ system 
via Front End Drivers (FED) devices; this information includes, for example, the BX 
of the collision from which the detected signal comes, the position
in the detector, etc. This information provided by several FEDs arrives to the 
Front-end Read-out Link (FRL) and finally they are delivered to the 
farm of processors through optical links, where the event information 
is processed by the HLT. 

\subsection{Data Processing}
\noindent Along this Chapter has been shown the process from the production of pp 
collisions by the LHC, the detection of the products of a collision by 
the CMS detector systems, up to the selection and storage of the interesting 
events by the Trigger and DAQ systems. However, there are three stages of data 
processing before they are used in a physics analysis. The selected events delivered 
by the HLT have a RAW data format, that contains the information coming from the 
detector but it is not suitable for analysis. Therefore, a reconstruction process is 
performed on the RAW data in order to identify all the physics objects and their 
kinematics in an event. CMS uses the Particle Flow algorithm in order to reconstruct 
every particle coming from the pp collision; the Particle Flow algorithm and the 
reconstruction of each particle are described in Chapter \ref{chap:ParticleID}. As a result, 
the reconstructed physics objects are stored in the RECO data format; however, the 
RECO data is still not appropriate for physics analysis due to the large size that they 
occupy on disk. Hence, two skimming processes are performed on the RECO data, resulting in 
a dataset with less size known as mini-AOD; the mini-AOD contains only the relevant information 
of the physics objects, making them appropriate for physics analysis.\\


%The full data are stored in pipelined-memories of processing elements, while waiting for the trigger decision.
%rigger latency, between a given bunch crossing and the distribution of the trigger decision to the
%detector front-end electronics, is 3.2 μs [91][14]. 

%The full data are stored in pipelined-memories of processing elements, while waiting for the trigger decision.
%rigger latency, between a given bunch crossing and the distribution of the trigger decision to the
%detector front-end electronics, is 3.2 μs [91][14]. 

%A schematic view of the components of the CMS DAQ system is shown in figure 9.2. The
%various sub-detector front-end systems (FES) store data continuously in 40-MHz pipelined buffers.
%Upon arrival of a synchronous L1 trigger (3.2 μs latency) via the Timing, Trigger and Control
%(TTC) system [204, 207], the corresponding data are extracted from the front-end buffers and
%pushed into the DAQ system by the Front-End Drivers (FEDs).





%Once the Trigger System identify an interesting event, a signal 
%is send to the read-out electronics in order 

%\subsection{Physics Objects}
%\label{subsec:PhysicsObjects}


