\chapter{Signal modeling}
\label{chap:sigmod}

In chapter \ref{chap:SM} of this work, the different possibilities of VLQ models implementations have been discussed. For the analysis presented in chapter~\ref{chap:search}, a specific model of VLQ's have been used for the MC simulations of a vector like \Tp. However, different models can be also studied, as shown in table~\ref{tab:VLQRepre}. Their predictions are similar, but the objects kinematics can be in principle different for the same production channel. The purpose of this appendix is to look in detail, at generator level, the kinematics of \Tp~and the associated jet in the single production mode for the different models. %With such study an estimation of the observability of the \Tp~with different models is done. %, the possibilities of the presented analysis to see a similar object to the one considered.

\section{Model Independence}
\label{sec:modindp}

There are three different possibilities to have a vector like \Tp, either as a singlet or as a member of a standard doublet ($Y=7/6$) or a non-standard doublet ($Y=1/6$). On the following, the singlet will be noted as T, the standard doublet as TB and the non-standard doublet as XT. The vector like \Tp~can be mixed to third or light generations of SM quarks exclusively, or can be mixed to the three generations. For the single production mode only the singlet representation can be mixed exclusively to third generation. The doublets should be mixed to the three generations or exclusively to light generations. These mixings are controlled by one single parameter for the MC generation, $R_{L}$. A $R_{L}=0$ stands for exclusive mixing to third generation, $R_{L}=\infty$ for exclusive mixing to light generations and $R_{L}=0.5$ for shared mixings to all three SM quark generations. 

On the plots that will be presented, the convention for the legend is \textit{Representation\_ $R_{L}$ value \_ Production channel}. For example, XT\_RLInf\_Tj stands for the XT doublet representation with $R_{L}=\infty$ in the single production of a vector like \Tp~associated to a jet. Figure~\ref{fig:accomjet} shows the distributions for the \pt~and $\eta$ of the jet produced in association with the \Tp~in the single production mode for all possible models and mixings. The comparison is performed with MC samples for a mass of the \Tp~of 700 \GeVcc, at parton level. No significant differences between models or mixings is observed for the majority of cases. However, the case of the singlet with exclusive mixing to third SM quark generation presents a more forward $\eta$ for the jet produced with the \Tp. Figure~\ref{fig:Tppt} displays the comparison between different models of the \pt~of \Tp. 

\begin{figure}[!hbtp]
  \begin{center}
    \includegraphics[width=0.45\textwidth]{figs/Ana/eta6thjetmodels.png}
    \includegraphics[width=0.45\textwidth]{figs/Ana/pt6thjetmodels.png}
    \caption{$\eta$ and $p_{T}$ of the associated jet produced with the \Tp~for different representations and couplings with SM quarks. The case of the singlet with exclusive mixing to third generation (T\_RL0\_Tj) produce a more forward associated jet than the rest of models.}
    \label{fig:accomjet}
  \end{center}
\end{figure}

\begin{figure}[!hbtp]
  \begin{center}
    \includegraphics[width=0.5\textwidth]{figs/Ana/ptTpmodels.png}
    \caption{$p_{T}$ of the \Tp~for different representations and couplings with SM quarks}
    \label{fig:Tppt}
  \end{center}
\end{figure}

In addition, as in the selection there is a cut on the $\Delta R$ between \Tp~and the associated jet, the distribution for this variable for different models can be found in figure~\ref{fig:DRmodels}. In this variable also no significant differences are seen between the majority of models. In this variable also the T\_RL0\_Tj case differs from others. The differences between the models should be diminished when considering hadronization and showering of MC samples.

\begin{figure}[!hbtp]
  \begin{center}
    \includegraphics[width=0.5\textwidth]{figs/Ana/DRmodels.png}
    \caption{$\Delta R (Tj)$ for different representations and couplings with SM quarks}
    \label{fig:DRmodels}
  \end{center}
\end{figure}

%We consider then as maximum a 5\% systematic uncertainty from differences shown between possible models, to take into account the theoretical predictions of kinematics of \Tp~and the associated jet in the single production mode.

\section{\Tjj~compared to \Tj}

The cross sections used as theoretical prediction were obtained considering only the main production process of \Tp~with only one associated jet. However, if the vector like \Tp~exists in nature it should be produced with extra jets added to the main process. This extra jet could be produced from NLO corrections or additional LO processes. For the present study only the additional LO processes were considered. In addition to the cross section prediction from these contributions, it is important to consider how kinematics of \Tp~and the associated jet changes when including an extra jet. For this purpose, an additional MC sample was produced for the same mass (M=700~\GeVcc) where the \Tj~and \Tjj~processes were produced simultaneously and then compared to the main sample, with only \Tj~process simulated.

The \Tj~production is done by an initial state of two quarks, as shown in figure~\ref{fig:ProdDiagSingle}. In contrast, \Tjj~production comes from quark-quark and quark-gluon initial states. Then, a significant contribution to the total production cross section is expected. An increment of 39\% was found with respect to \Tj~cross section, using MG to calculate this cross section. An schematic view of the additional quark-quark processes is shown in figure~\ref{fig:qqTjj}, and for quark-gluon processes in figure~\ref{fig:qgTjj}.

\begin{figure}[!hbtp]
  \begin{center}
    \includegraphics[scale=0.35]{figs/Ana/Tjj_qq_Tgq_1.jpg}
    \includegraphics[scale=0.35]{figs/Ana/Tjj_qq_Tgq_2.jpg}
    \caption{Schematic Feynman diagrams of the processes on the \Tjj~production with quark-quark initial state. All the solid lines are quarks and the curly line is a \Z/\W boson.}
    \label{fig:qqTjj}
  \end{center}
\end{figure}

\begin{figure}[!hbtp]
  \begin{center}
    \includegraphics[scale=0.7]{figs/Ana/Tjj_qg_Tqq_1.jpg}
    \includegraphics[scale=0.35]{figs/Ana/Tjj_qg_Tqq_2.jpg}
    \includegraphics[scale=0.35]{figs/Ana/Tjj_qg_Tqq_3.jpg}
    \caption{Schematic Feynman diagrams of the processes on the \Tjj~production with quark-gluon initial state. All the solid lines are quarks and the curly line is a \Z/\W boson.}
    \label{fig:qgTjj}
  \end{center}
\end{figure}

In figure~\ref{fig:6thJ_Tjj} the comparison of kinematics of the leading jet produced with \Tp~for \Tj~and \Tj+\Tjj~productions is shown. The $\eta$ distribution presented in this figure shows a forward-backward asymmetry. This behavior is coming from a limitation of the phase-space integrator of MG for t-channel processes. In addition, figure~\ref{fig:T_Tjj} displays the kinematics of \Tp~for the same cases. In it, the same forward-backward $\eta$ asymmetry is found. Also, a small fraction of events have been produced with a \Tp~with a $p_{T}<20$~GeV/c. These events correspond to the cases where the second jet produced with the \Tp~is coming from a gluon radiated from the \Tp, as seen in figure~\ref{fig:qqTjj}. No significant differences were found.

\begin{figure}[!hbtp]
  \begin{center}
    \includegraphics[width=0.45\textwidth]{figs/Ana/pt6thJ_Tjj.png}
    \includegraphics[width=0.45\textwidth]{figs/Ana/eta6thJ_Tjj.png}
    \caption{$p_{T}$ and $\eta$ of the leading jet produced with the \Tp~for inclusive \Tjj production and exclusive \Tj. The $\eta$ distribution shows a forward-backward asymmetry coming from a limitation of the phase-space integrator of MG for t-channel processes.}
    \label{fig:6thJ_Tjj}
  \end{center}
\end{figure}

\begin{figure}[!hbtp]
  \begin{center}
    \includegraphics[width=0.45\textwidth]{figs/Ana/ptT_Tjj.png}
    \includegraphics[width=0.45\textwidth]{figs/Ana/etaT_Tjj.png}
    \caption{$p_{T}$ and $\eta$ of the \Tp~for inclusive \Tjj~production and exclusive \Tj. The $\eta$ distribution shows a forward-backward asymmetry coming from a limitation of the phase-space integrator of MG for t-channel processes. The small fraction of events with a \Tp~with a $p_{T}<20$~GeV/c corresponds to the processes where the second jet produced with the \Tp~is coming from a gluon radiated from the \Tp, as seen in figure~\ref{fig:qqTjj}.}
    \label{fig:T_Tjj}
  \end{center}
\end{figure}

Furthermore, the distribution for $\Delta R (T'j)$ is presented in figure~\ref{fig:DR_Tjj}, where the jet is the leading one. For the inclusive \Tjj~there is a difference with \Tj~at low $\Delta R$ that represents 5\% of the whole distribution. This difference is coming from the events were the leading jet is the extra jet produced with the \Tj~main process.

\begin{figure}[!hbtp]
  \begin{center}
    \includegraphics[width=0.5\textwidth]{figs/Ana/DR_Tjj.png}
    \caption{$\Delta R (T'j)$ of the \Tp~for inclusive \Tjj~production and exclusive \Tj. The small fraction of events for the \Tjj~case with a $\Delta R (T'j)<3$ is coming from events where the leading jet is the extra jet produced with the main \Tj~process.}
    \label{fig:DR_Tjj}
  \end{center}
\end{figure}

A good sensitivity of the analysis presented in chapter~\ref{chap:search} is expected for the different models and \Tjj~production. Moreover, an important increase of the theoretical cross sections is expected when including \Tjj~production. However, to precisely estimate the impact of the observed differences in the observability of the \Tp, the same study should be done taking into account hadronization and detector simulation of signal samples. 
%We then expect a good sensitivity of the analysis selection to Tjj inclusive process. Moreover, taking into consideration hadronization and detector effects, selection efficiencies should be very close to the ones quoted in table~\ref{tab:cutflow}.